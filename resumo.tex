\chapter*{Agradecimientos}

A mi orientador, Profesor. Mg. Alexander Victor Ocsa Mamani, por estar siempre presente y atento. Por todos los años de orientación, de los trabajos de investigación que concluye esta tesis de maestría. Por los desafíos que me proporcionó, y principalmente por su confianza. A mis padres Julio y Hermelinda por los años de educación que me proporcionaron, mis hermanos Andy y Cristian, por todo el apoyo que siempre me han brindado, a mis abuelos Julio y Lidia por siempre haberme proporcionado cariño y apoyo. A Concytec y Cienciactiva por todo el apoyo y darnos la confianza de realizar investigación. A Fincyt por la investigación en el proyecto 217-FINCyT-IA-2013 MLearning: Middleware para la construcción de objetos de aprendizaje móviles en ambientes reactivos y adaptables al contexto, caso educación básica regular. A todos los profesores que nos proporcionaron los conocimientos necesarios para realizar investigación, y finalmente a todas las personas que directa o indirectamente contribuyeron para que llegase hasta aquí.





\chapter*{Resumen}
%\thispagestyle{empty}

\addcontentsline{toc}{section}{Resumen}

Los sistemas de recomendación en la actualidad nos ayudan a obtener resultados  de búsqueda cercano o adaptados a nuestras necesidades, en los últimos años este enfoque ha ido cambiando y se ha centrado en los sistemas e-Learning y dentro de lo que son los sistemas de gestión de aprendizaje, que son tecnologías educativas muy importante para el desarrollo académico de las estudiantes, dentro de los sistemas de recomendación tradicionales se hace un \emph{matching} entre lo que es las entradas del estudiante, que por lo general son las tareas y las notas que se le da al estudiante por su desenvolvimiento ante una tarea, que es proporcionada por un profesor o algún sistema.

Este trabajo propone un modelo de recomendación de contenidos educativos basado en el contexto de un usuario, el cual usa un modelo de contexto que incorpora el rol, las tareas, ejercicios de programación y su aplicación al problema de recomendación. Las recomendaciones se hacen sobre la base de la estimación de la diferencia  que existe entre el nivel de conocimiento actual de un usuario frente a las habilidades que requiere en su contexto que se encuentra. En el trabajo se usa una técnica de razonamiento probabilístico para las recomendaciones, para tener en cuenta las especificaciones inexactas de las competencias de los usuarios y los requerimientos en su contexto.

Los experimentos desarrollados en el contexto del estudiante, muestran que, usando un modelo de razonamiento probabilístico ayuda a obtener mejores recomendaciones de contenidos educativos, según a las competencias faltantes de un estudiante respecto a un tema que necesita aprender, lo cual se busca hacer una estandarización para sistemas de recomendación.




\singlespacing
\vspace*{0.5cm} \noindent \textbf{Palabras Clave:} Sistemas de recomendación, E-learning, Maquinas de aprendizaje, Redes bayesianas.

%nearest neighbor search, similarity search, random projection, complex and unstructured data, motif-discovery.



\chapter*{Abstract}
%\thispagestyle{empty}

\addcontentsline{toc}{section}{Abstract}
The Recommender systems today help us to obtain results close search adapted to our needs, in recent years this approach ah been changing and has focused on e-Learning systems and within what are learning management  systems , which are educational technologies very important for the academic development of students  within traditional systems of recommendation makes a \emph {matching} between what is the inputs of the student, who usually are the tasks and notes given to the student for his development to a task, which is provided by a teacher or some system.

This project proposes a recommendation model for educational content  based on the context of a user, which uses a context model that incorporates the role, tasks, programming exercises and their application to the problem of recommendation.The  Recommendations are made on the basis of the estimate of the difference between the current level of knowledge of a user in front of the skills required in their work context.This project work a technique of probabilistic reasoning is used for recommendations to take into account the incorrect specifications of user skills and requirements in their  context.

The experiments developed in the context of the student, show that, using a model of probabilistic reasoning helps to get better recommendations of educational content, according to the missing competences of a student on an issue that needs to learn, which seeks to standardization for recommendation systems.

\singlespacing
\vspace*{0.5cm} \noindent \textbf{Keywords:} Recommendation system, E-learning, Machine Learning, Bayesian network.

