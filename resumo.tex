\chapter*{Agradecimientos}

A mi orientador, Profesor. Dr. Cristian Lopez del Alamo, por estar siempre presente y atento. Por todos los años de orientación, de los trabajos de investigación que concluye esta tesis de doctorado. Por los desafíos que me proporcionó, y principalmente por su confianza. A mis padres y a toda mi familia por los años de educación que me proporcionaron. A mi  novia  Karla, por todo el apoyo que siempre me ha brindado. A Concytec y Cienciactiva por todo el apoyo y darnos la confianza de realizar investigación.  A todos los profesores que nos proporcionaron los conocimientos necesarios para realizar investigación, y finalmente a todas las personas que directa o indirectamente contribuyeron para que llegase hasta aquí.





\chapter*{Resumen}
%\thispagestyle{empty}

\addcontentsline{toc}{section}{Resumen}

La creciente disponibilidad de datos en diferente dominio de aplicación ha motivado el desarrollo de técnicas de recuperación y descubrimiento de conocimiento en grandes volúmenes de datos.   Recientes trabajos muestran que tanto las técnicas de aprendizaje profundo como  nuevos métodos de búsqueda aproximada en dominio de datos complejos son campos de investigación importantes, donde tanto la eficiencia como la  escalabilidad de los algoritmos son factores críticos. Para resolver el problema de escalabilidad se han propuesto muchos enfoques. En problemas de gran escala con datos en altas dimensiones, una solución de búsqueda aproximada con un análisis teórico solido se muestra más adecuado que una solución exacta con un modelo teórico débil.    Algoritmos de búsqueda aproximada  basados en \textit{hashing} son propuestos para consultar en conjuntos de datos  alta dimensiones debido a su velocidad de recuperación y bajo costo de almacenamiento.  Por otro lado, en problemas donde se tiene grandes volúmenes de datos etiquetados las técnicas de aprendizaje profundo, como las redes convolucionales, se muestran más adecuadas conforme el número de ejemplos por clases crece.

Estudios recientes, promueven el uso de la \acf{CNN} con técnicas de  \textit{hashing} para mejorar la precisión de la búsqueda de los k-vecinos más cercanos - KNN.  Sin embargo, aun hay retos que resolver para encontrar una solución práctica y eficiente para indexar características  CNN, tales como la necesidad de un proceso de entrenamiento intenso para lograr resultados de consulta precisos y la dependencia crítica de los parámetros.   Con el fin de superar estos problemas, se propone un nuevo método de búsqueda por similitud, \textit{Deep frActal based  Hashing (DAsH)}, para calcular los mejores valores de parámetros  para una proyección óptima en un subespacio, explorando las correlaciones entre los atributos de las características  CNN usando la teoría fractal. Además, inspirado por recientes avances  en redes CNN, utilizamos no solo activaciones de capas inferiores que son más generales, sino también el conocimiento previo de los datos semánticos sobre la última capa CNN para mejorar la precisión de la búsqueda.  Así, nuestro método produce una mejor representación del espacio de datos con un coste computacional menor para una mejor precisión. Esta mejora  significativa en velocidad y precisión nos permite evaluar este esquema en conjuntos de datos reales y sintéticos.



\singlespacing
\vspace*{0.5cm} \noindent \textbf{Palabras Clave:}  Recuperación de Información, kNN, Deep Learning, Dimensión Fractal, Hashing.

 
 