%%%%%%%%%%%%%%%%%%%%%%%%%%%%%%%%%%%%%%%%%%%%%%%%%%%%%%%%%%%%%%%%%%%%%%%%%%
%   This is frontpage.tex file needed for the dmathesis.cls file.  You   %
%  have to  put this file in the same directory with your thesis files.  %
%                Written by M. Imran 2001/06/18                          % 
%                 No Copyright for this file                             % 
%                 Save your time and enjoy it                            % 
%                                                                        % 
%%%%%%%%%%%%%%%%%%%%%%%%%%%%%%%%%%%%%%%%%%%%%%%%%%%%%%%%%%%%%%%%%%%%%%%%%%%
%%%%%%%%%%%%%%%%%%%%%%%%%%%%%%%%%%%%%%%%%%%%%%%%%%%%%%%%%%%%%%%%%%%%%%%%%%%
%%%%%%%%%%%%%%%%           The title page           %%%%%%%%%%%%%%%%%%%%%%%  
%%%%%%%%%%%%%%%%%%%%%%%%%%%%%%%%%%%%%%%%%%%%%%%%%%%%%%%%%%%%%%%%%%%%%%%%%%%
\pagenumbering{roman}
%\pagenumbering{arabic}

\setcounter{page}{1}

\newpage

\thispagestyle{empty}
\begin{center}
  \vspace*{1cm}
 {\Large\bf Métodos semánticos para la recuperación de información en grandes volúmenes de datos: Una arquitectura escalable y eficiente}


  \vspace*{2cm}
  {\Large\bf Alexander Victor Ocsa Mamani}

  \vfill

  {\Large A Thesis presented for the degree of\\
         [1mm] Doctor of Philosophy}
  \vspace*{0.9cm}
  
  % Put your university logo here if you wish.
   \begin{center}
   \includegraphics{DU_2-col_sml.eps}
   \end{center}

  {\large Your Research Group Here\\
          [-3mm] Department of Mathematical Sciences\\
          [-3mm] University of Durham\\
          [-3mm] England\\
          [1mm]  Month and Year}

\end{center}

%%%%%%%%%%%%%%%%%%%%%%%%%%%%%%%%%%%%%%%%%%%%%%%%%%%%%%%%%%%%%%%%%%%%%%%%%%%
%%%%%%%%%%%%%%%% The dedication page, of you have one  %%%%%%%%%%%%%%%%%%%%  
%%%%%%%%%%%%%%%%%%%%%%%%%%%%%%%%%%%%%%%%%%%%%%%%%%%%%%%%%%%%%%%%%%%%%%%%%%%
\newpage
\thispagestyle{empty}
\begin{center}
 \vspace*{2cm}
  \textit{\LARGE {Dedicated to}}\\ 
 Someone here
\end{center}


%%%%%%%%%%%%%%%%%%%%%%%%%%%%%%%%%%%%%%%%%%%%%%%%%%%%%%%%%%%%%%%%%%%%%%%%%%%
%%%%%%%%%%%%%%%%%%           The abstract page         %%%%%%%%%%%%%%%%%%%%  
%%%%%%%%%%%%%%%%%%%%%%%%%%%%%%%%%%%%%%%%%%%%%%%%%%%%%%%%%%%%%%%%%%%%%%%%%%%
\newpage
\thispagestyle{empty}
\addcontentsline{toc}{chapter}{\numberline{}Abstract}
\begin{center}
  \textbf{\large Abstract}
\end{center}


La creciente disponibilidad de datos en diferente dominio de aplicación ha motivado el desarrollo de técnicas de recuperación y descubrimiento de conocimiento en grandes volúmenes de datos.   Recientes trabajos muestran que tanto las técnicas de aprendizaje profundo como  nuevos métodos de búsqueda aproximada en dominio de datos complejos son campos de investigación importantes, donde tanto la eficiencia como la  escalabilidad de los algoritmos son factores críticos. Para resolver el problema de escalabilidad se han propuesto muchos enfoques. En problemas de gran escala con datos en altas dimensiones, una solución de búsqueda aproximada con un análisis teórico solido se muestra más adecuado que una solución exacta con un modelo teórico débil.    Algoritmos de búsqueda aproximada  basados en \textit{hashing} son propuestos para consultar en conjuntos de datos  alta dimensiones debido a su velocidad de recuperación y bajo costo de almacenamiento.  Por otro lado, en problemas donde se tiene grandes volúmenes de datos etiquetados las técnicas de aprendizaje profundo, como las redes convolucionales, se muestran más adecuadas conforme el número de ejemplos por clases crece.

Estudios recientes, promueven el uso de la \acf{CNN} con técnicas de  \textit{hashing} para mejorar la precisión de la búsqueda de los k-vecinos más cercanos - KNN.  Sin embargo, aun hay retos que resolver para encontrar una solución práctica y eficiente para indexar características  CNN, tales como la necesidad de un proceso de entrenamiento intenso para lograr resultados de consulta precisos y la dependencia crítica de los parámetros.   Con el fin de superar estos problemas, se propone un nuevo método de búsqueda por similitud, \textit{Deep frActal based  Hashing (DAsH)}, para calcular los mejores valores de parámetros  para una proyección óptima en un subespacio, explorando las correlaciones entre los atributos de las características  CNN usando la teoría fractal. Además, inspirado por recientes avances  en redes CNN, utilizamos no solo activaciones de capas inferiores que son más generales, sino también el conocimiento previo de los datos semánticos sobre la última capa CNN para mejorar la precisión de la búsqueda.  Así, nuestro método produce una mejor representación del espacio de datos con un coste computacional menor para una mejor precisión. Esta ganancia significativa en velocidad y precisión nos permite evaluar este esquema en grandes,  realistas y desafiantes  configuraciones de conjuntos de datos.


%%%%%%%%%%%%%%%%%%%%%%%%%%%%%%%%%%%%%%%%%%%%%%%%%%%%%%%%%%%%%%%%%%%%%%%%%%%
%%%%%%%%%%%%%%%%%%          The declaration page         %%%%%%%%%%%%%%%%%%  
%%%%%%%%%%%%%%%%%%%%%%%%%%%%%%%%%%%%%%%%%%%%%%%%%%%%%%%%%%%%%%%%%%%%%%%%%%%
\chapter*{Declaration}
\addcontentsline{toc}{chapter}{\numberline{}Declaration}
The work in this thesis is based on research carried out at the
YOUR RESEARCH GROUP HERE, the Department of Mathematical Sciences, England. No part of this thesis has been submitted elsewhere for any other degree or qualification and it is all
my own work unless referenced to the contrary in the text. [If your thesis based on joint research , you have to mention what part of it is your individual constribution, see Rules for the Submission of Work for Higher Degrees for detail. You will get one from the Graduate School.]



\vspace{2in}
\noindent \textbf{Copyright \copyright\; 2001 by YOUR NAME HERE}.\\
``The copyright of this thesis rests with the author.  No quotations
from it should be published without the author's prior written consent
and information derived from it should be acknowledged''.



%%%%%%%%%%%%%%%%%%%%%%%%%%%%%%%%%%%%%%%%%%%%%%%%%%%%%%%%%%%%%%%%%%%%%%%%%%%
%%%%%%%%%%%%%%%%%%     The acknowledgements page         %%%%%%%%%%%%%%%%%%  
%%%%%%%%%%%%%%%%%%%%%%%%%%%%%%%%%%%%%%%%%%%%%%%%%%%%%%%%%%%%%%%%%%%%%%%%%%%
\chapter*{Acknowledgements}
\addcontentsline{toc}{chapter}{\numberline{}Acknowledgements}
Thank to someone who prepared this template. Thank to someone who
prepared this template. Thank to someone who prepared this template.
Thank to someone who prepared this template. Thank to someone who
prepared this template. Thank to someone who prepared this template.
Thank to someone who prepared this template. Thank to someone who
prepared this template. Thank to someone who prepared this
template.Thank to someone who prepared this template. Thank to someone
who prepared this template. Thank to someone who prepared this
template. Thank to someone who prepared this template. Thank to
someone who prepared this template. Thank to someone who prepared this
template. Thank to someone who prepared this template. Thank to
someone who prepared this template.

%%%%%%%%%%%%%%%%%%%%%%%%%%%%%%%%%%%%%%%%%%%%%%%%%%%%%%%%%%%%%%%%%%%%%%%%%%%
%%%%%%%%    tableofcontents, listoffigures and listoftables       %%%%%%%%%
%%%%%%%%        Command if you do not have  them                  %%%%%%%%%
%%%%%%%%%%%%%%%%%%%%%%%%%%%%%%%%%%%%%%%%%%%%%%%%%%%%%%%%%%%%%%%%%%%%%%%%%%%
\tableofcontents
\listoffigures
\listoftables
\clearpage


%%%%%%%%%%%%%%%%%%%%%%%%%%%%%%%%%%%%%%%%%%%%%%%%%%%%%%%%%%%%%%%%%%%%%%%%%%%
%%%%%%%%%%%%%%%%%%%%%%   END OF FRONT PAGE %%%%%%%%%%%%%%%%%%%%%%%%%%%%%%%%
%%%%%%%%%%%%%%%%%%%%%%%%%%%%%%%%%%%%%%%%%%%%%%%%%%%%%%%%%%%%%%%%%%%%%%%%%%%









